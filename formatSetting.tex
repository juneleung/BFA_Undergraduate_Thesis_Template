% Author: Juneleung
% Date  : 2023-08-01
% Contact : Juneleungchan@163.com
% Github: https://github.com/juneleung/BFA_Undergraduate_Thesis_Template

\usepackage[left=3.17cm, right=3.17cm, top=2.54cm, bottom=2.54cm]{geometry} %页边距
% \CTEXsetup[format={\Large\bfseries}]{section} %设置章标题字号为Large,居左
%\CTEXsetup[number={\chinese{section}}]{section}
%\CTEXsetup[name={(,)}]{subsection}
%\CTEXsetup[number={\chinese{subsection}}]{subsection}
%\CTEXsetup[name={(,)}]{subsubsection}
%\CTEXsetup[number=\arabic{subsubsection}]{subsubsection}  %以上四行为各级标题样式设置,可根据需要做修改
%\linespread{1.5} %设置全文行间距
 
%\usepackage[english]{babel}
%\usepackage{float}     %放弃美学排版图表
\usepackage{ctex}   
\usepackage{fontspec}   %修改字体.
\usepackage{amsmath, amsfonts, amssymb} % 数学公式相关宏包
\usepackage{color}      % color content
\usepackage{graphicx}   % 导入图片
\usepackage{subfigure}  % 并排子图
\usepackage{url}        % 超链接
\usepackage{bm}         % 加粗部分公式,比如\bm{aaa}aaa
\usepackage{multirow}
\usepackage{booktabs}
\usepackage{epstopdf}
\usepackage{epsfig}
\usepackage{longtable}  %长表格
\usepackage{supertabular}%跨页表格
\usepackage{algorithm}
\usepackage{algorithmic}
\usepackage{changepage}
\usepackage{times}
\usepackage{titlesec}
\usepackage{array}
\usepackage{setspace}
\usepackage{listings}
\usepackage{xcolor}
\usepackage{bm}
\usepackage{tikz}
\usepackage{mathpazo}
\usepackage{pgfplots}
\usepackage{url}
\usepackage{pifont}
\usepackage[perpage]{footmisc}
\usepackage{microtype}
\usepackage{scrextend}
\usepackage{cite}
\usepackage{titletoc}
\usepackage{fancyhdr}
\usepackage{tabularx}
\usepackage{lineno}
\usepackage{indentfirst}

\pagestyle{plain}
\fancyfoot[C]{\thepage}

% -- Code Format --
\definecolor{CPPLight}  {HTML} {686868}
\definecolor{CPPSteel}  {HTML} {888888}
\definecolor{CPPDark}   {HTML} {262626}
\definecolor{CPPBlue}   {HTML} {4172A3}
\definecolor{CPPGreen}  {HTML} {487818}
\definecolor{CPPBrown}  {HTML} {A07040}
\definecolor{CPPRed}    {HTML} {AD4D3A}
\definecolor{CPPViolet} {HTML} {7040A0}
\definecolor{CPPGray}  {HTML} {B8B8B8}
\lstset{
    columns=fixed,       
    numbers=left,  %  none,left                          % 在左侧显示行号
    frame=none,                                          % 不显示背景边框
    backgroundcolor=\color[RGB]{245,245,244},            % 设定背景颜色
    keywordstyle=\color[RGB]{40,40,255},                 % 设定关键字颜色
    numberstyle=\footnotesize\color[RGB]{255,0,0},       % 设定行号格式
    commentstyle=\it\color[RGB]{0,96,96},                % 设置代码注释的格式
    stringstyle=\rmfamily\slshape\color[RGB]{128,0,0},   % 设置字符串格式
    showstringspaces=false,                              % 不显示字符串中的空格
    language=c++,                                        % 设置语言
    morekeywords={alignas,continute,friend,register,true,alignof,decltype,goto,
    reinterpret_cast,try,asm,defult,if,return,typedef,auto,delete,inline,short,
    typeid,bool,do,int,signed,typename,break,double,long,sizeof,union,case,
    openvdb,FloatGrid,Ptr,Coord,Accessor,FloatTree,math,Transform,
    dynamic_cast,mutable,static,unsigned,catch,else,namespace,static_assert,using,
    char,enum,new,static_cast,virtual,char16_t,char32_t,explict,noexcept,struct,
    void,export,nullptr,switch,volatile,class,extern,operator,template,wchar_t,
    const,false,private,this,while,constexpr,float,protected,thread_local,
    const_cast,for,public,throw,std},
    emph={map,set,multimap,multiset,unordered_map,unordered_set,
    unordered_multiset,unordered_multimap,vector,string,list,deque,
    array,stack,forwared_list,iostream,memory,shared_ptr,unique_ptr,
    random,bitset,ostream,istream,cout,cin,endl,move,default_random_engine,
    uniform_int_distribution,iterator,algorithm,functional,bing,numeric,},
    emphstyle=\color{CPPViolet}, 
}

% -- CHN Font --
% mac 
% \setCJKmainfont{Songti SC}   % 宋体
% \setCJKfamilyfont{song}{Songti SC}[AutoFakeBold]
% \setCJKfamilyfont{kai}{Kaiti SC}[AutoFakeBold]
% \setCJKfamilyfont{hei}{STHeiti}
% \setCJKfamilyfont{fsong}{STFangsong}[AutoFakeBold]
% \newcommand{\song}{\CJKfamily{song}}
% \newcommand{\kai}{\CJKfamily{kai}}
% \newcommand{\hei}{\CJKfamily{hei}}
% \newcommand{\fsong}{\CJKfamily{fsong}}

\setCJKfamilyfont{song}[Path={font/},AutoFakeBold]{simsun.ttc}
\newcommand{\song}{\CJKfamily{song}} 

\setCJKfamilyfont{kai}[Path={font/},AutoFakeBold]{simkai.ttf}
\newcommand{\kai}{\CJKfamily{kai}} 

\setCJKfamilyfont{hei}[Path={font/},AutoFakeBold]{simhei.ttf}
\newcommand{\hei}{\CJKfamily{hei}} 

\setCJKfamilyfont{fsong}[Path={font/},AutoFakeBold]{simfang.ttf}
\newcommand{\fsong}{\CJKfamily{fsong}} 


% -- ENG Font --
\setmainfont{Times New Roman}
\newcommand{\TimesNR}{Times New Roman}

% -- 摘要 目录 --
\renewcommand\contentsname{\zihao{-2}\hei{目\qquad 录}}
% \renewcommand\listfigurename{插图目录}
% \renewcommand\listtablename{表格目录}
% \renewcommand\refname{参考文献}
% \renewcommand\indexname{索引}
% \renewcommand\figurename{图}
% \renewcommand\tablename{表}
% \renewcommand\abstractname{摘要}
% \renewcommand\partname{部分}
% \renewcommand\appendixname{附录}
% \renewcommand\today{\number\year年\number\month月\number\day日}
\renewcommand\algorithm{算法}


% -- Footnote Format --
\renewcommand{\footnotesize}{\fontsize{9pt}{11pt}\selectfont}

% -- Cite Format --
\newcommand{\upcite}[1]{\textsuperscript{\textsuperscript{\cite{#1}}}}



% -- ContentTable Format --
\titlecontents{section}
        [1em]
        {\zihao{4}\hei}
        {\contentslabel{1em}}
        {\hspace*{-1em}}
        {~\titlerule*[0.6pc]{$.$}~\contentspage}

\titlecontents{subsection}
        [3em]
        {\zihao{4}\hei}
        {\contentslabel{1.5em}}
        {\hspace*{-1.5em}}
        {~\titlerule*[0.6pc]{$.$}~\contentspage}

\titlecontents{subsubsection}
        [5em]
        {\zihao{-4}\hei}
        {\contentslabel{2.5em}}
        {\hspace*{-2.5em}}
        {~\titlerule*[0.6pc]{$.$}~\contentspage}


% -- Section Format --
\CTEXsetup[name={,.}]{section}
\titleformat*{\section}{\hei\zihao{3}\centering}
\titleformat*{\subsection}{\hei\zihao{-3}}
\titleformat*{\subsubsection}{\hei\zihao{4}}
\titleformat*{\paragraph}{\song}



\renewcommand{\thefootnote}{\ding{\numexpr171+\value{footnote}}}
\deffootnote[1.5em]{1.5em}{1em}{\thefootnotemark\space}
 
